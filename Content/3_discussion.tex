The problems of ethics, morality and autonomous robots has seen much discussion
from several different points of view. For the topic of autonomous vehicles I
will break the problem down into the actors and different kinds of possible scenarios.

The actors I consider involved in autonomous vehicles are:
\begin{itemize}
\item The passengers of the vehicle;
\item The autonomous vehicle;
\item The company programming, constructing and selling the vehicle;
\item And the government that set the legal framework allowing the vehicles.
\end{itemize}

I will then discuss the moral responsibility for each of these actors in the
following scenarios:
\begin{enumerate}
\item A scenario where the vehicle has the potential to cause an accident, but
  the possibility of avoiding it if acting correctly;
\item A scenario where the vehicle must several different wrong possible actions
  --- all of which will yield an accident of some sort.
\end{enumerate}

This rather general scenarios will be discussed more in general and specific
variations of them provided to further exemplify.

\subsection{The passengers}
\begin{itemize}
\item the passenger takes one decision: choosing to use an autonomous vehicle,
  all consequences derive from that action.
\item in the first scenario it seems not
  to be the fault of the passenger if the car crash, indeed the scenario could
  be said to occur constantly as the vehicle always has the possibility of doing
  a dumb action. So if there is an action that is correct, the passenger's
  decision to use an autonomous vehicle does not seem to be at fault.
\item in the second scenario the passenger could be held more responsible as the car does
  the best of the situation and indeed it could be argued that the accident
  would not have happened had the passenger not chosen to use the autonomous
  vehicle.
\item however, it still seems a bit odd that the passenger is held
  responsible for the actions of a vehicle if the passenger has no way to affect
  them.
\end{itemize}
\subsection{The autonomous vehicle}
\begin{itemize}
\item the vehicle could be considered to be an actor taking part in the scenario as
it is in practice the one taking the actions
\item to say that the vehicle is morally responsible for an action is to by
extension consider the vehicle capable of morality. Without going into a
discussion of morality and robots, it does not seem as if the robots we will see
in 5 years are sufficiently autonomous as to be capable of morality. And if
indeed we say they are, we got more complex moral problems in using them to
drive us around
\end{itemize}

\subsection{The company}
\begin{itemize}
\item The company will be treated as one single actor without regard for
  individuals in the company such as the developer who wrote the bug or similar,
  this as the company can be considered responsible for the final product and so
  by extension the actions of its employees.
\item The company producing the car could be seen as to be the one most clearly
  at blame when an autonomous vehicle crash, as they are the ones that have
  programmed and built the car.
\item However, it is important to consider the fact that we know already now
  that these traffic accidents will occur. So in allowing the vehicles we have
  to some extent also accepted these traffic accidents.
\item in the first scenario it seems that the responsibility can be put on the
  company for having created a car that in a scenario where the car could
  reasonably have avoided the error didn't do so. The reason it didn't must be
  due to a fault in how it is built
\item in the second scenario it does on the other hand seem odd that the company
  is to blame for an accident that could not have been avoided. If the car did
  what could be considered to be the best possible of it, then why are the
  company responsible?
\end{itemize}

\subsection{The government}
\begin{itemize}
\item the government here refers to some authority that allowed the autonomous
  vehicle on the road and set the legal framework for what might be considered a
  legal autonomous vehicle.
\item in scenario 1 it seems that the government is not to be held responsible
  for the cars action, while they allowed it on the road they probably did so
  under the assumption that it would be able to avoid traffic accidents that
  could have been reasonably avoided.
\item in scenario 2 it does seem as if the government is to be held at least a
  little responsible, they knew that the vehicles would cause accidents and
  still allowed them --- by extension the government allowed the accident to happen.
\end{itemize}
