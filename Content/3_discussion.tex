The problems of ethics, morality and autonomous robots has seen much discussion
from several different points of view. For the topic of autonomous vehicles I
will break the problem down into the actors and different kinds of possible scenarios.

The actors I consider involved in autonomous vehicles are:
\begin{itemize}
\item The passengers of the vehicle;
\item The autonomous vehicle;
\item The company programming, constructing and selling the vehicle;
\item And the government that set the legal framework allowing the vehicles.
\end{itemize}

I will then discuss the moral responsibility for each of these actors in the
following scenarios:
\begin{itemize}
\item A scenario where the vehicle has the potential to cause an accident, but
  the possibility of avoiding it if acting correctly;
\item A scenario where the vehicle must several different wrong possible actions
  --- all of which will yield an accident of some sort.
\end{itemize}

This rather general scenarios will be discussed more in general and specific
variations of them provided to further exemplify.

\subsection{The passengers}
the passengers

\subsection{The autonomous vehicle}
the autonomous vehicle

\subsection{The company}
the company

\subsection{The government}
the government