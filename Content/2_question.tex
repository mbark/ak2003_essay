This essay will attempt to find an answer to the following question:
\textit{Who is morally responsible for an autonomous car?}
In order to do some distinctions have to be done.

First of all, when an entity is referred to as being morally responsible for an
autonomous car it means that the the entity is responsible for the morally
significant actions performed by the car. This definition follows the one
defined by Eshleman in~\cite{eshleman_2014_moral_mra} who further illustrate
what that entails by writing the following:
``Thus, to be morally responsible for something, say an action, is to be
worthy of a particular kind of reaction --- praise, blame, or something akin to
these --- for having performed it.''

When discussing the term autonomous car is used deliberately in
order to describe a car that is fully autonomous, requiring nothing from any
passenger in the car. The actions of the car are therefore decided by the car
alone without any outside input.

The concept of moral responsibility will be split up into two more distinct
terms: casually responsible and morally blameworthy. The former, casually
responsible, refers to being the actor that takes responsibility for the actions
of the autonomous car. The latter, morally blameworthy, means being to blame ---
in moral terms --- for the actions of the autonomous car. To illustrate the
difference between the two, consider a scenario where the car crashes and causes
damage to a property to save five persons' lives. The manufacturer might be
considered casually responsible for the action, but seeing as the car saved five
lives they are not be morally blamed for the action taken by the car.

The car is considered not to be an moral agents; without specifying the criteria
for what it means to be a moral agent it seems unlikely that the autonomous cars
coming out in the coming years will fulfill even the most basic of criteria. As
such, the possibility is not taken into account in the analysis.

Finally, the introduction of autonomous cars is considered to be good
and something to strive for. It is assumed that traffic accidents will decrease
and that there will be a positive gain from introducing the cars. This is not
yet proven to be true~\cite{schoettle_2015_preliminary_apaorcisv}, but discussing
the case where they are worse than human drivers is considerably less
interesting (see~\cite{marchant_2012_coming_ccbavatlst} for further motivation).