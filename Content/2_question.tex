This essay will attempt to find an answer to the following question:
\textit{Who is morally responsible for an autonomous car?}
In order to do some distinctions have to be done.

First of all, when an entity is referred to as being morally responsible for an
autonomous car it means that the the entity is responsible for the morally
significant actions performed by the car. This definition follows the one
defined by Eshleman in~\cite{esheleman_2014_moral_mra} who further illustrate
what that entails by writing the following:
``Thus, to be morally responsible for something, say an action, is to be
worthy of a particular kind of reaction --- praise, blame, or something akin to
these --- for having performed it.''
While this definition suffers from some problems, they are not of any major
concern for this discussion.

Furthermore, when discussing the term autonomous car is used deliberately in
order to describe a car that is fully autonomous, requiring nothing from any
passenger in the car. The actions of the car are therefore decided by the car
alone without outside input being possible.
