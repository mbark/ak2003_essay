In order to analyze who is morally responsible for an autonomous car I will
go through all actors that might be potentially considered responsible ---
discussing what the consequences of that actor being responsible would mean in
terms of duties etc.

The actors that might be considered responsible for an autonomous car are:
\begin{enumerate}
\item The passenger or passengers of the vehicle;
\item The manufacturer responsible for programming, constructing and selling the car;
\item And the government that set the legal framework allowing the vehicles.
\end{enumerate}

For each of these actors an analysis of the moral responsibility for that actor
in two different scenarios will be discussed. The two scenarios are described below:
\begin{enumerate}
\item\label{item:scenario1} \textit{The avoidable scenario} --- where the vehicle has the
  potential to cause an accident, but the possibility of avoiding it if acting
  correctly;
\item\label{item:scenario2} \textit{The unavoidable scenario} --- where the
  vehicle must choose between several different wrong possible actions --- all
  of which will yield an accident of some sort.
\end{enumerate}

Scenario~\ref{item:scenario2} might be considered an example of the Trolley problem, a
scenario where an actor must choose between several outcomes none of which is the
obviously correct or incorrect one~\cite{2016_trolley_tp}.

The analysis will be done from two perspectives:
\begin{itemize}
\item A deontological perspective based on the idea that ``ought implies
  can'', which extends to the fact that moral responsibility derives from an
  ability to affect the outcome (see~\cite{stern_2015_kantian_kevaao}[Ch. 7]
  for a more in-depth discussion);
\item And a consequentialistic perspective where the focus is on what the
  positive consequences will be of holding the actor morally responsible.
\end{itemize}

The following three sections will therefore analyze each actor in turn,
considering the two scenarios described above.

\subsection{The passenger}
For the sake of this analysis it is considered to be only one passenger in the
car; the difference between one or many passengers is of minor relevance to the
analysis.

For the sake of this discussion it is important to consider what decisions the
passenger made that might be considered to influence the outcome in the two
scenarios. The most obvious decision was to choose to use an autonomous car, had
the passenger not used the car none of the two scenarios would have happened.

However, a less obvious decision made by the passenger is also that of choosing
car. If, as seems likely, there are several competing cars with different
implementations and thus different ways of acting --- the passenger has made a
deliberate decision regarding the actions that the car will take in traffic.

\subsubsection{The avoidable scenario}\label{sec:passenger:avoidable}
From a deontological perspective the passenger can be held morally blameworthy
for those actions that they could affect. As mentioned above, these are choosing
to use an autonomous car and choosing which car to buy and use.

Hevelke et.\ al.\ point to the fact that using a car will lead to an increased
risk of an accident, thus making the passenger to at least a small degree
blameworthy~\cite{hevelke_2014_responsibility_rfcoavaea}. From this they
conclude that the passenger is to a small degree blameworthy, although it is not
the individual passenger's fault --- it was just bad luck the accident occurred
--- but rather all user's of autonomous cars.

Another thing to take into account is if the passenger deliberately chose a
cheaper and less safe car in which case the passenger could affect the outcome
to some extent, adding blameworthiness to them.

Holding the passenger responsible for the actions of the car would potentially
have positive effects in reducing car usage, while not restricting it ---
hopefully reducing traffic accidents. As such, from both points of views
the passenger is --- at least to a small degree --- morally blameworthy.

\subsubsection{The unavoidable scenario}
In the case where the car can not avoid a crash and instead must make a moral
decision the blameworthiness of the outcome can to some extent be considered to
be the passenger's. If the passenger was presented two cars, one implementing
Consequentialism and one Deontology, the moral action taken by the car is just
an extension of the passenger's choice of morality.

Once again, it is important to point out that if the passenger chose between two
cars without knowing the morality of either, the consequences were out of the
passenger's control and thus they can not be held morally blameworthy.

For a consequentialist in the case where the scenario is unavoidable holding the
passenger responsible would lead to an increased responsibility when buying a
car and choosing how it should act. Given that sufficient information exists so
that the passenger can make an educated decision, this might improve the
passenger's feeling of responsibility and awareness of the issue.

\subsection{The manufacturer}\label{sec:manufacturer}
The manufacturer is considered as one single actor, no distinction is made between
individuals in the company or other manufacturer's that provide parts to
manufacturer.

The manufacturer is the actor taking the most actions of all actors involved as
it constructs, designs and programs the car thus being in control of almost
every aspect of it.

\subsubsection{The avoidable scenario}
Intuition would probably tell us that the manufacturer is to blame for an
autonomous car that makes a mistake seeing as they are the ones to built the car
and thus the ones to introduce the code leading to the faulty decision process.
This is an argument that relies on the deontological way of thinking that the
manufacturer's were those that could affect the outcome, thus those that
ought to have fixed the problem.

However, as Goodall says:
``Any system ever engineered has occasionally failed.''~\cite{goodall_2014_machine_meaav}.
It is therefore rather a question of whether the company chose to not be
thorough, thus putting the blame on them. Or if they were as thorough as might
be considered reasonable, in which case they are not morally blameworthy as they
did what they ought to do.

Once again the Ford Pinto case (mentioned in
Section~\ref{sec:passenger:avoidable}) is similar in that disregarding safety
without informating makes the company morally blameworthy. If the company
informs the public, they shift the moral blameworthiness to the passenger.

This conclusion agrees with what a consequentialistic view might arrive at as
well: companies disregarding safety should be held morally blameworthy for doing
so, but if they do \textit{the best possible} holding them morally blameworthy
would discourage further development of the cars.

\subsubsection{The unavoidable scenario}
From a deontological perspective it seems hard to argue that the manufacturer is
to blame --- they did what they could. This is also supported the fact that
holding manufacturers responsible may deter them from developing or increase the
prices of the cars, neither of which is advantageous~\cite{marchant_2012_coming_ccbavatlst}.

Once again a consequentialist point of view would agree that holding the
manufacturer responsible would only discourage development of autonomous cars,
while not providing seemingly very many positive benefits.

\subsection{The government}
The government refers to the authority that allowed the autonomous car on the
road and set the legal framework for what is considered a legal autonomous car.

Once again we need to establish what decisions the government made; they allowed
autonomous cars on the road and they defined what criteria an autonomous car
need to fulfill in order for it to be allowed.

\subsubsection{The avoidable scenario}
In this scenario it seems that the moral blameworthiness can be put on the
government if the autonomous car is found to be legal, yet still acts
incorrectly. In that case, the government could reasonably have done more, and
thus ought to have done so. Thus, they are in that case at least partially
morally blameworthy.

If the car on the other hand was illegal it becomes a matter of enforcing the
law --- did the government do what they could to ensure that all cars on the
road abide the law? If they do not, the government has at least partial moral
blameworthiness.

This also agrees with that a consequentialist would argue; if we do not hold the
government blameworthy to some extent then the companies will continue to build
bad cars, causing more accidents and deaths.

\subsubsection{The unavoidable scenario}
In the case where the crash is unavoidable some blame can be put on the
government as they allowed the car on the road, by extension allowing the
accident to happen. However, consider the cars will save lives and reduce the
number of accidents, so in that regard the government did what they ought to ---
they allowed autonomous cars. The moral blameworthiness therefore seems hard to
put on the government in this case as they did what they ought to do --- which
is to try to prevent lives.

Additionally there is little to gain from holding the government morally
blameworthy, the only real consequence it could have would be that the cars
become illegal and that would mean an increase in accidents and deaths.
