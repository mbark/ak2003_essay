The analysis will be done from two different points of view:
\begin{itemize}
\item A \textit{deontological} perspective based on the idea that ``ought implies
  can'', which extends to the fact that moral responsibility derives from an
  ability to affect the outcome (see~\cite[Ch. 7]{stern_2015_kantian_kevaao}
  for a more in-depth discussion);
\item And a \textit{consequentialistic} perspective based the idea that the actor who
  should be held morally responsible is the one for which doing so will provide
  the best consequences.
\end{itemize}

In order to analyze who is morally responsible for an autonomous car I will
go through all actors that might be potentially considered responsible ---
discussing what the consequences of that actor being responsible would mean from
the two perspectives outlined.

The actors that might be considered responsible for an autonomous car are:
\begin{enumerate}
\item The passenger or passengers of the vehicle;
\item The manufacturer responsible for programming, constructing and selling the car;
\item And the government setting the legal framework, allowing the vehicles.
\end{enumerate}

For each of these actors an analysis of the moral responsibility for that actor
in two different scenarios will be discussed. The two scenarios are described below:
\begin{enumerate}
\item\label{item:scenario1} \textit{The avoidable scenario} --- where the
  vehicle crashes, but had the possibility of avoiding it if it had acted
  correctly;
\item\label{item:scenario2} \textit{The unavoidable scenario} --- where the
  vehicle crashes and had to choose between several different possible
  actions --- all of which would have yielded an accident of some sort (consider
  it as a variant of the famous Trolley problem~\cite{2016_trolley_tp})
\end{enumerate}

The following three sections will analyze each actor in turn, considering the
two scenarios described above.

\subsection{The passenger}\label{sec:passenger}
For the sake of this analysis it is assumed to be only one passenger in the
car; the difference between one or many passengers is of minor relevance for the
analysis.

The actions the passenger made that might be considered to have influenced the
outcome in the scenarios are:
\begin{itemize}
\item Choosing to use an autonomous car, had the passenger not done so there
  would have been no possibility for either of the two scenarios to occur;
\item And choosing which car to use between --- as seems likely to be the case
  --- several competing models.
\end{itemize}

\subsubsection{The avoidable scenario}
The action of choosing to use an autonomous car is one that greatly affects the
outcome as if the passenger had not chosen to do so, no crash would have
occurred. It therefore seems that the passenger should be held to some degree
blameworthy for the crash as they could affect the outcome. However, as Hevelke
et.\ al.\ point out, it is not the individual passenger's fault that the car
crashed, they did not do anything worse than any other passenger in an
autonomous car~\cite{hevelke_2014_responsibility_rfcoavaea}. As a consequence it
is not the individual passenger that is to be held blameworthy, but rather all
user's of autonomous cars that are to be held to some degree blameworthy for all
autonomous cars.

The second action, choosing which car between several competing models also is
one that could bear with it some moral blameworthiness. Consider that the user
had the ability to buy a more expensive and safe car but chose the less safe one
involved in the crash. In this case, the passenger could clearly affect the
outcome to some extent, adding blameworthiness to them.

From a consequentialistic point of view it seems that holding the passenger to
some degree responsible for the actions of the car could have possible
consequences in reducing car usage in favor of other methods of transportation.
Compared to the current responsibility held by the driver, who is almost fully
responsible for a car, it would still mean a reduced responsibility and a
motivation to use an autonomous car rather than a normal one.

\subsubsection{The unavoidable scenario}
In the case where the car can not avoid a crash and instead must make a moral
decision the blameworthiness of the outcome can to some extent be considered to
be the passenger's. If the passenger was presented two cars, one implementing
Consequentialism and one Deontology, the moral action taken by the car is just
an extension of the passenger's choice of morality.

Once again, it is important to point out that if the passenger chose between two
cars without knowing the morality of either, the consequences were out of the
passenger's control and thus they can not be held morally blameworthy.

For a consequentialist in the case where the scenario is unavoidable holding the
passenger responsible would lead to an increased responsibility when buying a
car and choosing how it should act. Given that sufficient information exists so
that the passenger can make an educated decision, this might improve the
passenger's feeling of responsibility and awareness of the issue.

\subsection{The manufacturer}\label{sec:manufacturer}
The manufacturer is considered as one single actor, no distinction is made between
individuals in the company or other manufacturer's that provide parts to
manufacturer.

The manufacturer is the actor that could be said to ``perform the majority of
the actions'' out of the three actors considered, the reason is that the
manufacturer designs, constructs and programs the car --- putting them in
control of almost every aspect of it.

\subsubsection{The avoidable scenario}
It seems intuitive that the manufacturer should in the avoidable scenario be
held responsible, they are the ones that built the car and thus the ones that
introduced --- or didn't fix --- the error that caused the car to not avoid the
crash.

However, as Goodall points out: ``Any system ever engineered has occasionally
failed.''~\cite{goodall_2014_machine_meaav}. It should therefore rather be
discussed whether the company were sufficiently thorough in regards to safety.
If they are found to be \textit{reasonably thorough} they did what they could
and as such are not morally blameworthy.

A good example of this might be the Ford Pinto
case~\cite[Ch.3]{vandepoel_2011_ethics_etaeai} where Ford could have avoided a
three people dying in a car crash if they had spent \$11 more dollars. It seems
that they reasonably could have done more to prevent the accident and thus ought
to have done so.

This does not necessarily mean the manufacturer must make their car as expensive
as possible to introduce all safety concerns, they might also inform the
consumers of the safety risks and allow them to make the choice instead. In
doing so the manufacturer respects the autonomy of the consumer and shifts moral
blameworthiness to them instead (as also discussed in Section~\ref{sec:passenger})

This conclusion agrees with what a consequentialistic view might arrive at as
well: companies disregarding safety should be held morally blameworthy for doing
so as it would force them to improve the safety of the cars, thus reducing
accidents. If a company however are found to have done \textit{the best
  possible}, holding them morally blameworthy would only discourage
manufacturers from making autonomous cars --- which would have negative
consequences.

\subsubsection{The unavoidable scenario}
From a deontological perspective it seems hard to argue that the manufacturer is
to blame --- they did what they could.

From a consequentialistic point of view holding the manufacturer responsible
even though they did what they could would improve nothing and only deter them
from making autonomous cars~\cite{marchant_2012_coming_ccbavatlst}.

\subsection{The government}
The government refers to the authority that allowed the autonomous car on the
road and set the legal framework for what is considered a legal autonomous car.

We need to establish what actions has made that influenced the outcome:
\begin{itemize}
\item They defined what standards an autonomous car need to uphold to be
  allowed on the roads;
\item And they chose to which extent these standards are controlled and enforced.
\end{itemize}

\subsubsection{The avoidable scenario}
If the autonomous car acts incorrectly even though it is legal it seems as if
the government could reasonably have done more by setting stricter laws. Thus,
they are at least partially morally blameworthy.

If the car on the other hand was illegal, it becomes a matter of enforcing the
law --- did the government do what they could have reasonably done to ensure
that all cars on the road abide the law? If they did not, the government has at
least partial moral blameworthiness.

If we assume the companies to act only according to the law and not take
morality into consideration, it must the government's job to ensure that the law
and morality are as close as possible. Thus, a consequentialistic point of view
agree that the government need to be held responsible if the laws or the
enforcement of them is insufficient --- otherwise, the cars will see no
improvement.

\subsubsection{The unavoidable scenario}
In the case where the crash is unavoidable it could be argued that some blame
could be put on the government, when allowing the cars they by extension allowed
the accident to happen. However, the basis for this discussion is that the cars
are better than human drivers. This means that the government did what they
could to reduce traffic accidents, they allowed the autonomous cars. As they did
what they ought to do, they are not to be held morally blameworthy.

Neither does there seem to be any positive consequences of holding the
government responsible in this scenario. The only consequence that seems to come
from holding them responsible is complete ban of all cars --- which is clearly
not positive.
