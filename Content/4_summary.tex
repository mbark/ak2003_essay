To summarize the analysis given above I will once again return to the two
scenarios and summarize how the blame is divided between the actors. Finally, I
give a short conclusion that should answer the question posed at the start of
this essay:
\textit{Who is morally responsible for an autonomous car?}

\subsection{The avoidable scenario}
In this scenario I have argued that all user's of autonomous cars are to be held
morally responsible for the actions of the cars. Furthermore, if the passenger
has made an informed decision when selecting the car to use they also take some
moral responsibility.

The manufacturer is blameworthy if disregarded safety or failed to inform the
consumer of the risks involved in using their autonomous car. Otherwise, they
are not morally blameworthy.

Finally, the government is morally blameworthy if it is the case that the car
was legal and still acted incorrectly, or illegal and the law insufficiently
enforced.

As a result, the passenger will always be to some degree morally blameworthy for
the actions of the car whereas the manufacturer and government can avoid moral
blame if they are able to provide proof that they have done what could have bean
reasonably asked of them.

\subsection{The unavoidable scenario}
In this scenario the passenger is to a small degree morally responsible for the
car if the passengers choice of car was deliberate and informed and included
deciding how the car should act morally in an unavoidable scenario.

I argued that the manufacturer is not to be held morally blameworthy in this
scenario as they did what they could and it would provide nothing but negative
consequences to hold them responsible.

Finally, I argued that the government acted correctly given the basic assumption
underlying this discussion --- that the cars will reduce accidents and save
lives. Apart from this there also seem to be no positive consequences to holding
the government responsible.

As a result, the moral responsibility for the car in this scenario can only to a
small degree be put on the passenger. The manufacturer and government are
neither morally blameworthy. This seems to some extent intuitive; in a scenario
with a negative outcome, which every actor did their best to prevent, no one is
to be blamed.

\subsection{Conclusion}
To conclude, who is morally responsible for an autonomous car depends on the
actions taken by the actors involved and the responsibility is usually split
between one or more of them.

For the avoidable scenario, the moral responsibility is split between all actors
involved, with the possibility of the manufacturer and government to motivate
that they performed their duties to a sufficient degree as to not be held
morally blameworthy.

For the unavoidable scenario the responsibility can be put on the passenger if
the choice of car was also a choice of how a car should act morally. The
manufacturer and government are found to not be morally responsible.